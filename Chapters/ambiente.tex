%************************************************
\chapter{Ambiente di lavoro}\label{ch:ambiente}
%************************************************
Lo svolgimento del progetto è stato effettuato su dispositivi personali in quanto l'azienda ospitante ha ritenuto che ciò potesse offrire un vantaggio consentendomi di preparare l'ambiente di lavoro in modo autonomo, imparando a gestire tutti i software richiesti e risolvendo le eventuali problematiche che avrei potuto incontrare.

Il sistema operativo di riferimento all'interno dell'azienda è \emph{Microsoft Windows 7} ma per problemi di connessione alla rete aziendale ho deciso di utilizzare un sistema operativo GNU/Linux.

\section{Hardware}
\begin{description}
\item[notebook:] Asus K73sv
	\begin{description}
	\item[sistema operativo:] Kubuntu 13.04
	\item[utilizzo:] progettazione, sviluppo e test applicazioni
	\end{description}
\item[tablet:] Asus ME301T
	\begin{description}
	\item[sistema operativo:] Android 4.2.1
	\item[utilizzo:] installazione e test applicazioni
	\end{description}
\end{description}

\section{Sencha Touch 2}
\begin{figure}[htb]
\centering
\includegraphics[scale=0.4]{gfx/sencha-large}
\caption{Logo Sencha}
\label{fig:sencha}
\end{figure}

\begin{description}
\item[versioni:] 2.2.1 e 2.2.3
\end{description}

\emph{Sencha Touch} è un framework JavaScript che consente di sviluppare web application per dispositivi mobili sfruttando tecnologie web standard come \acs{HTML5} e \acs{CSS3}; l'obiettivo che ha è quello di riprodurre il look and feel delle applicazioni native per i diversi sistemi operativi mobili e generare applicazioni multi-piattaforma, consentendo ad uno sviluppatore di scrivere una sola volta il codice per poi compilarlo per ogni piattaforma desiderata.

Attualmente \emph{Sencha Touch} supporta i sgeuenti sistemi operativi:
\begin{itemize}
\item Android
\item iOS
\end{itemize}

\section{Sencha Cmd}
\begin{description}
\item[versioni:] 3.1.2 e 4.0.0
\end{description}

\emph{Sencha Cmd} è un tool a riga di comando che fornisce diverse azioni automatizzate durante tutto il ciclo di vita delle applicazioni, dalla generazione di un nuovo progetto con la relativa struttura di cartelle e file principali, all'aggiunta di modelli e controller, fino alla compilazione e al deployment.

Esso viene utilizzato congiuntamente a \emph{Sencha Touch} e nella versione più recente consente l'utilizzo di \emph{Apache Cordova} per aggiungere importanti caratteristiche alle applicazioni.
 
\section{Apache Cordova - Adobe PhoneGap}
\begin{figure}[htb]
\centering
\includegraphics[scale=0.15]{gfx/cordova_logo_normal_dark}
\caption{Logo Apache Cordova}
\label{fig: logo cordova}
\end{figure}

\begin{description}
\item[versioni:] 3.0.0 e 3.1.0
\end{description}

\emph{Apache Cordova} è una piattaforma di sviluppo per applicazioni native per dispositivi mobili che usa gli standard web \acs{HTML5}, \acs{CSS3} e JavaScript.

Fornisce un set di \ac{API} che permette ad uno sviluppatore di accedere alle funzionalità native dei device come la fotocamera o l'accelerometro tramite l'utilizzo di JavaScript; non fornisce però nessuna libreria grafica che consenta di creare \ac{GUI} ma combinato con Sencha Touch consente la creazione di applicazioni native multi-piattaforma con un look and feel accattivante.

Attualmente supporta i seguenti sistemi operativi:
\begin{itemize}
\item Android
\item iOS
\item Blackberry
\item Windows Phone 7
\item Windows Phone 8
\item Windows 8
\item Firefox OS
\end{itemize}

\begin{figure}[htb]
\centering
\includegraphics[scale=0.25]{gfx/PhoneGap}
\caption{Logo Adobe PhoneGap}
\label{fig: logo phonegap}
\end{figure}

\emph{PhoneGap} è il progetto originario del framework il cui codice è stato donato ad \ac{ASF} con lo scopo di rendere lo sviluppo e la manutenzione più trasparenti e di rendere più agevoli i contributi da parte di altre organizzazioni.

Ad oggi esso può essere considerato come una distribuzione di \emph{Cordova}, dal quale eredita le \ac{API} ma al quale aggiunge alcuni importanti caratteristiche; la più rilevante è il sistema \emph{PhoneGap Build} \cite{adobe:phonegapBuild} il quale consente di compilare remotamente il codice della propria applicazione per le seguenti piattaforme:
\begin{itemize}
\item Android
\item iOS
\item BlackBerry 6
\item Windows Phone 7
\item WebOS
\item Symbian
\end{itemize}

La compilazione remota consente ad uno sviluppatore di ottenere una build della propria applicazione per ogni piattaforma disponibile senza la necessità di possedere sistemi operativi e \ac{SDK} dedicati.
Questo servizio può essere utilizzato in tre modi diversi:
\begin{itemize}
\item Caricando il progetto da compilare direttamente nel sito dedicato;
\item Creando un link tra l'account \emph{GitHub} e l'account di \emph{PhoneGap Build};
\item Creando un link tra la \emph{PhoneGap \ac{CLI}} e l'account di \emph{PhoneGap Build}.
\end{itemize}

La possibilità di ottenere applicazioni funzionanti per ogni sistema operativo mobile disponibile è però vincolata dal possesso delle chiavi e delle licenze necessarie fornite dai vari produttori.

\subsubsection{Apache Cordova CLI - Adobe PhoneGap CLI}
La \emph{Cordova \ac{CLI}} è uno strumento a riga di comando che consente la creazione di un progetto con la relativa struttura di cartelle e file, la compilazione per le diverse piattaforme e l'esecuzione tramite un emulatore.

Affinché tale tool sia in grado di eseguire il processo di build per la piattaforma desiderata è necessario aver installato sul proprio sistema il \ac{SDK} appropriato, il che richiede anche un sistema operativo specifico.

Quest'ultima considerazione non è valida per il sistema di build di \emph{PhoneGap}, in quanto consente di richiamare dalla \ac{CLI} un tool di compilazione remota, previa registrazione al servizio cloud nel sito dedicato.

\section{Sencha Architect}
\begin{description}
\item[versioni: 2.2.2 e 2.2.3]
\end{description}

\emph{Sencha Architect} è lo strumento ufficiale \emph{Sencha} per la creazione di interfacce grafiche sfruttando le \ac{API} di \emph{Sencha Touch} oppure di \emph{Ext JS}\footnote{Framework JavaScript per lo  sviluppo di desktop web application.}.

Tale strumento si è dimostrato molto utile nella creazione di interfacce grafiche complesse, velocizzando notevolmente l'aggiunta dei diversi contenitori ed elementi grafici da inserire al loro interno.

Oltre all'interfaccia grafica è inoltre possibile implementare la logica di gestione di un'applicazione in uno o più controller oppure sfruttando singoli handler per ogni evento scatenato.

\section{Android SDK}
\emph{Android \ac{SDK}} fornisce le \ac{API} e gli strumenti necessari per sviluppare, testare e compilare applicazioni per Android.

È composto da pacchetti modulari, installabili e aggiornabili separatamente tramite l'apposito manager, tra i quali possiamo trovare:
\begin{description}
\item[SDK tools:] contiene strumenti per effettuare il debug e i test delle applicazioni;
\item[SDK platform-tools:] contiene strumenti dipendenti dalla piattaforma per effettuare lo sviluppo e il debug delle applicazioni;
\item[SDK platform:] ogni versione di Android possiede una piattaforma specifica che include il file \code{android.jar} con la libreria completa;
\item[system images:] ogni piattaforma fornisce una o più immagini di sistema necessarie per l'utilizzo dell'emulatore;
\item[sources:] contiene una copia dei sorgenti della piattaforma.
\end{description}

\section{Eclipse}
\begin{description}
\item[versione: 4.3.1 Kepler]
\end{description}

\emph{Eclipse} è un \ac{IDE} multi-linguaggio e multi-piattaforma costituito da uno spazio di lavoro di base e da numerosi plugin aggiuntivi, installabili separatamente, che forniscono funzionalità supplementari al sistema.

Il set di base fornisce gli strumenti necessari allo sviluppo di applicazioni Java ma tramite plugin è possibile ampliare i linguaggi supportati: infatti per sviluppare applicazioni Android è necessario installare il plugin \ac{ADT} che fornisce una serie di strumenti che aiutano un programmatore nello sviluppo.

\section{JSDuck}
\emph{JSDuck} è uno strumento che permette di generare la documentazione del codice utilizzando i commenti opportunamente strutturati con l'utilizzo di tag appropriati.

Utilizza \emph{Markdown} \cite{daringFireball:markdown} per convertire il testo dei commenti in codice \ac{HTML} e costruire un documento ipertestuale navigabile offline che contiene, in modo ordinato e facilmente consultabile, tutta la documentazione dell'applicazione.