%************************************************
\chapter{Ambiente di lavoro}\label{ch:ambiente}
%************************************************
Lo svolgimento del progetto è stato effettuato su dispositivi personali in quanto l'azienda ospitante ha ritenuto che ciò potesse offrire un vantaggio consentendomi di preparare l'ambiente di lavoro in modo autonomo, imparando a gestire tutti i software richiesti e risolvendo le eventuali problematiche che avrei potuto incontrare.

Il sistema operativo di riferimento all'interno dell'azienda era \emph{Microsoft Windows 7} ma per problemi di connessione alla rete aziendale ho deciso di utilizzare un sistema operativo GNU/Linux.

\section{Hardware}
\begin{description}
\item[notebook:] Asus K73sv
	\begin{description}
	\item[sistema operativo:] Kubuntu 13.04
	\item[utilizzo:] progettazione, sviluppo e test applicazioni
	\end{description}
\item[tablet:] Asus ME301T
	\begin{description}
	\item[sistema operativo:] Android 4.2.1
	\item[utilizzo:] installazione e test applicazioni
	\end{description}
\end{description}

\section{Sencha Touch 2}
\begin{figure}[htb]
\centering
\includegraphics[scale=0.4]{gfx/sencha-large}
\caption{Logo Sencha}
\label{fig:sencha}
\end{figure}

\begin{description}
\item[versioni:] 2.2.1 e 2.2.3
\end{description}

Sencha Touch è un framework Javascript che consente di sviluppare web application per dispositivi mobili sfruttando tecnologie web standard come \acs{HTML5} e \acs{CSS3}; l'obiettivo che ha è quello di riprodurre il look and feel delle applicazioni native per i diversi sistemi operativi mobili e generare applicazioni multipiattaforma, consentendo ad uno sviluppatore di scrivere una sola volta il codice per poi compilarlo per ogni piattaforma desiderata.

Attualmente Sencha Touch supporta i sgeuenti sistemi operativi:
\begin{itemize}
\item Android
\item iOS
\end{itemize}

\section{Sencha Cmd}
\begin{description}
\item[versioni:] 3.1.2 e 4.0.0
\end{description}

È un tool a riga di comando che fornisce diverse azioni automatizzate durante tutto il ciclo di vita delle applicazioni, dalla generazione di un nuovo progetto con la relativa struttura di cartelle e file principali, all'aggiunta di modelli e controller, fino alla compilazione e al deployment.

Esso viene utilizzato congiuntamente a Sencha Touch e nella versione più recente consente l'utilizzo di Apache Cordova per aggiungere importanti caratteristiche alle applicazioni.
 
\section{Apache Cordova - PhoneGap}
\begin{figure}[htb]
\centering
\subfloat[][Logo Apache Cordova]
	{\includegraphics[scale=0.15]{gfx/cordova_logo_normal_dark}} \quad
\subfloat[][Logo PhoneGap]
	{\includegraphics[scale=0.25]{gfx/PhoneGap}}
\caption{Loghi Cordova / PhoneGap}
\label{fig:cordova phonegap}
\end{figure}

\begin{description}
\item[versioni:] 3.0.0 e 3.1.0
\end{description}

Apache Cordova è una piattaforma di sviluppo per applicazioni native per dispositivi mobili che usa gli standard web \acs{HTML5}, \acs{CSS3} e Javascript.

Fornisce un set di \ac{API} che permette ad uno sviluppatore di accedere alle funzionalità native dei device come la fotocamera o l'accelerometro tramite l'utilizzo di Javascript; combinato con Sencha Touch consente la creazione di applicazioni native per diversi sistemi operativi.

Attualmente supporta i seguenti sistemi operativi:
\begin{itemize}
\item Android
\item iOS
\item Blackberry
\item Windows Phone
\item Palm WebOS
\item Bada
\item Symbian
\end{itemize}

\subsubsection{Apache Cordova CLI}
La Command-line Interface è uno strumento a riga di comando che consente la creazione di un progetto con la relativa struttura di cartelle e file, la compilazione per le diverse piattaforme e l'esecuzione tramite un emulatore.

Affinché tale tool sia in grado di eseguire il processo di build per la piattaforma desiderata è necessario aver installato sul proprio sistema il \ac{SDK} appropriato, il che richiede anche un sistema operativo specifico.

\section{Sencha Architect}

\section{Android Developer Tool}

\section{Eclipse}

\section{Weinre}

\section{Jasmine}

\section{JSDuck}