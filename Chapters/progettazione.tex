%************************************************
\chapter{Progettazione}\label{ch:progettazione}
%************************************************
Il seguente capitolo descrive le scelte progettuali che sono state effettuate, ovvero viene presentata l'architettura del sistema attraverso diagrammi delle classi e vengono illustrate le interazioni dell'utente col sistema attraverso i diagrammi di attività.

\subsection{Sencha Touch 2}
JavaScript, il linguaggio con cui è stato scritto il framework Sencha Touch, è un linguaggio di scripting prototype-based e multi-paradigma, supporta infatti sia la programmazione imperativa e object-oriented che quella funzionale.
Tali caratteristiche lo rendono un linguaggio molto flessibile che permette di risolvere lo stesso problema in molti modi e con differenti tecniche; purtroppo la mancanza di struttura nel linguaggio porta il grande peso dell'imprevedibilità.


\subsection{Descrizione architettura}

\subsection{Diagrammi di attività}