%************************************************
\chapter{Pianificazione}\label{ch:pianificazione}
%************************************************
Lo svolgimento dello stage è stato suddiviso in 3 macro-attività distinte derivanti dalle necessità dell'azienda ospitante di studiare l'utilizzo di 2 framework alternativi e di applicare tali conoscenze in 2 applicazioni che presentano finalità diverse; ogni attività è poi stata suddivisa in sotto-attività per caratterizzare meglio il progetto da effettuare.
\begin{description}
\item[Introduzione all'ambiente di lavoro:] \hfill
	\begin{itemize}
	\item Studio framework Sencha Touch 2.2.1;
	\item Studio componente \emph{SyncEngine};
	\item Progettazione,sviluppo e test applicazione \emph{MyNotes}.
	\item Realizzazione documentazione;
	\end{itemize}
\item[Gestione delle periferiche:] \hfill
	\begin{itemize}
	\item Studio framework Apache Cordova (Adobe PhoneGap);
	\item Studio interfacciamento con periferiche del dispositivo;
	\item Progettazione, sviluppo e test applicazione \emph{SensorDevice}.
	\item Realizzazione documentazione;
	\end{itemize}
\item[Utilizzo Sencha Architect e Sencha Touch 2.3:] \hfill
	\begin{itemize}
	\item Studio e utilizzo Sencha Architect 2.2.x;
	\item Studio framework Sencha Touch 2.3;
	\item Implementazione e test nuove feature su applicazione \emph{SensorDevice};
	\item Stesura documento di sintesi dei risultati ottenuti.
	\end{itemize}
\end{description}
Le attività di maggior peso sono le prime 2 in quanto richiedono un profondo studio degli strumenti di lavoro nonché un'attenta fase di progettazione e sviluppo dei 2 prototipi.

La terza attività non era compresa nella pianificazione iniziale dello stage, ma è stata inserita in corso d'opera in quanto è stata rilasciata la nuova versione del framework Sencha Touch e, viste le interessanti caratteristiche offerte, si è ritenuto, di comune accordo con l'azienda, di dedicare l'ultimo periodo di stage allo studio delle nuove feature introdotte.

\section{Ciclo di vita}
A causa della natura sperimentale dello stage si è scelto di utilizzare un ciclo di vita incrementale in modo da consentire uno sviluppo graduale dei prototipi fornendo 