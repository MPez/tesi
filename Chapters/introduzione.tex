%************************************************
\chapter{Introduzione}\label{ch:introduzione}
%************************************************

\section{Azienda ospitante}
L'azienda \myCompanyExtended, nata nel 1986 con sede a Padova e uffici operativi a Pescara e Udine, opera nel settore della gestione aziendale delle \acs{PMI} da oltre 25 anni; utilizza tecnologie avanzate e consolidate per sviluppare e distribuire prodotti software applicativi completi, flessibili, integrati, aperti ad una costante innovazione e strutturati per essere facilmente implementati nelle variegate realtà aziendali.

Fornisce inoltre servizi professionali di consulenza e supporto tecnico e si affianca a partner riconosciuti e affidabili quali:
\begin{description}
	\item[SAP:] leader mondiale nelle soluzioni software per il business, fornisce applicazioni e servizi per supportare aziende di qualsiasi dimensione operanti in più di 25 differenti settori di mercato;
	\item[IBM]: società di innovazione al servizio delle aziende e delle istituzioni di tutto il mondo che detiene primati in ogni area tecnologica, dai microprocessori ai supercomputer, dai server al software per lo sviluppo e la gestione di complesse infrastrutture informatiche;
	\item[Oracle]: azienda leader mondiale nello sviluppo e distribuzione di sistemi informatici hardware ed enterprise software, in particolare il proprio \ac{DBMS}.
	L'azienda costruisce anche strumenti per lo sviluppo di database e sistemi di software middle-tier, software di pianificazione delle risorse aziendali (\ac{ERP}), software di gestione delle relazioni con i clienti (\ac{CRM}) e software di gestione della catena di distribuzione (\ac{SCM}).
	\item[Able Tech:] società di Ricerca e Sviluppo specializzata esclusivamente in soluzioni di Document e Process management. Il successo e la crescita continua di AbleTech S.r.l, azienda da dieci anni sul mercato Italiano, è legato alla soluzione di Business Process Management \emph{ARXivar}: document e content management, workflow e process management, archiviazione ottica e conservazione sostitutiva, tutto in un'unica soluzione.
\end{description}


\section{Descrizione dello stage}
Nello scenario mondiale la presenza di dispositivi mobili quali smartphone e tablet è sempre più pervasiva, ogni giorno interagiamo con decine di applicazioni che ci aiutano a svolgere le attività più diverse; anche il mondo del lavoro ha capito la potenzialità di questi dispositivi e la politica \ac{BYOD} delle aziende fa si che ogni lavoratore possa disporre di un terminale per svolgere le proprie attività aziendali.

L'azienda \myCompany sta intraprendendo questa strada in quanto ha la necessità di fornire ai propri rappresentanti un modo per effettuare la propria attività di raccolta ordini presso i clienti accedendo facilmente ai dati dell'impresa anche all'esterno del confine aziendale.

Attualmente l'azienda utilizza una web application sviluppata a questo scopo il che rende molto limitato l'uso che se ne può fare: infatti tale applicazione può funzionare solo online e questo non sempre è possibile a causa dell'eventuale mancanza di rete; è necessario al contrario che un prodotto del genere dia la possibilità ai propri utilizzatori di non avere vincoli di connettività per lavorare, la soluzione a questo problema è una applicazione nativa per sistemi operativi mobili che dia la possibilità di essere utilizzata offline e, non appena sia disponibile una connessione, proceda con il download o l'upload dei dati.

Proprio per questo motivo l'azienda ha deciso di proporre 2 stage correlati per approfondire tali tematiche attraverso l'utilizzo di 2 framework che consentono lo sviluppo di applicazioni multipiattaforma: \emph{Sencha Touch 2} e \emph{Apache Cordova (PhoneGap)}.


\subsection{Obiettivi}
Lo scopo del primo stage è stato di sviluppare un sistema di archiviazione dati offline e di sincronizzazione online con un server dedicato: il prodotto di tale progetto è stato il modulo \emph{SyncEngine}.

Lo stage da me affrontato è la seconda parte del suddetto progetto; esso si compone di 2 attività principali ed è quindi divisibile in 2 parti distinte con obiettivi specifici:

\paragraph*{Studio e utilizzo SyncEngine}
Il primo obiettivo è di utilizzare questo componente all'interno di un applicazione di gestione note, capendone il funzionamento e cercando possibili criticità e, nel caso esistano, proponendo una soluzione ad esse.

Questo modulo fornisce infatti una classe manager sviluppata come un'interfaccia con la quale utilizzare tutto il sistema senza conoscerne i dettagli ed il funzionamento preciso.

\paragraph*{Studio interfacciamento sensoristica}
La seconda parte del progetto prevede lo studio per l'accesso alle \ac{API} native di sistemi operativi mobili fornite da Sencha Touch e da Apache Cordova.

I 2 framework forniscono metodi di accesso ai device e funzionalità diverse, ma possono essere utilizzati congiuntamente: lo scopo è quindi sviluppare una o più applicazioni che sfruttino queste potenzialità, fornendo in questo modo all'azienda un resoconto dettagliato sulle funzionalità realmente utilizzabili.


\section{Scopo del documento}
Questo documento intende descrivere dettagliatamente tutto il processo che ha portato al soddisfacimento degli obiettivi pianificati, dalle fasi iniziali del progetto fino alla raccolta dei risultati ottenuti dallo sviluppo delle applicazioni.

Il documento è così strutturato:
\begin{description}
\item[secondo capitolo:] descrive la pianificazione temporale dello stage suddivisa per singole attività; 
\item[terzo capitolo:] descrive l'ambiente di lavoro utilizzato per la progettazione, lo sviluppo e i test delle applicazioni;
\item[quarto capitolo:] sono raccolti i casi d'uso e i requisiti rilevati;
\item[quinto capitolo:] descrive l'architettura dei diversi prototipi progettati;
\item[sesto capitolo:] descrive come sono stati sviluppati i prototipi e quali problemi sono stati affrontati;
\item[settimo capitolo:] descrive la metodologia di test applicata al progetto;
\item[ottavo capitolo:] raccoglie le considerazioni conclusive dello studio effettuato durante il periodo di stage.
\end{description}
\subsection{Notazioni}