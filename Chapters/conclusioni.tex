%************************************************
\chapter{Conclusioni}\label{ch:conclusioni}
%************************************************ 
\section{Risultati}
Il progetto di stage ha prodotto due prototipi che soddisfano tutti i requisiti individuati durante l'attività di analisi; l'azienda era interessata a testare le applicazioni anche su dispositivi basati su \emph{iOS} ma ciò non è stato possibile a causa dei ritardi nell'ottenimento della licenza necessaria per effettuare il processo di build per tale piattaforma.

In particolare è stato possibile integrare efficacemente il modulo \emph{SyncEngine} e testarne il corretto funzionamento all'interno della semplice applicazione \emph{MyNotes} di gestione di note ed autori.

Per quanto riguarda l'applicazione \emph{SensorDevice}, la tabella \ref{tab:funzionalità sensordevice} illustra il comportamento dei framework utilizzati con le funzionalità che l'azienda desiderava testare: l'unica soluzione capace di utilizzare tutte le periferiche presenti in un dispositivo mobile è stata \emph{ApacheCordova}, i cui plugin sono risultati essere di facile utilizzo e di immediata comprensione grazie alla bontà della documentazione.

\begin{table}[htb]
\caption{Resoconto funzionalità testate -- SensorDevice}
\label{tab:funzionalità sensordevice}
\centering
\begin{tabular}{cccc}
\hiderowcolors
\toprule
\multirow{2}*{Funzionalità} & \multicolumn{2}{c}{Sencha Touch} 			& \multirow{2}*{Apache Cordova} \\
\cmidrule(rl) {2-3} 
							& 2.2.1 		& 2.3.0						& \\
\midrule %\showrowcolors
Barcode						& $\circ$ 		& $\circ$					& $\checkmark$ \\
Camera						& $\checkmark$ 	& $\checkmark$				& $\checkmark$ \\
Capture						& $\circ$ 		& $\bullet$					& $\checkmark$ \\
Connection					& $\checkmark$ 	& $\checkmark$				& $\checkmark$ \\
Contacts					& $\bullet$ 	& $\checkmark$				& $\checkmark$ \\
Device						& $\checkmark$ 	& $\checkmark$				& $\checkmark$ \\
File						 	& $\circ$ 		& $\bullet$					& $\checkmark$ \\
Geolocation					& $\checkmark$ 	& $\checkmark$ 				& $\checkmark$ \\					
\bottomrule
\end{tabular}
\begin{quotation}\footnotesize
\item[$\circ$:] non presente
\item[$\bullet$:] presente ma non funzionante
\item[$\checkmark$:] funzionante
\end{quotation}
\end{table}

\emph{Sencha Touch 2} si è rilevato un'ottima soluzione per realizzare l'interfaccia grafica grazie ai numerosi automatismi del framework e all'architettura \ac{MVC} che semplifica la gestione e la separazione delle diverse unità software; purtroppo è risultata essere ancora carente con l'implementazione di un sistema capace di interagire correttamente con i sensori dei device.

La situazione in proposito è migliorata notevolmente con la distribuzione della versione \emph{2.3} del framework ma l'impressione è che il sistema sia ancora immaturo sotto questo punto di vista e con ampi margini di miglioramento.

Un altro aspetto importante da considerare è l'utilizzo di \emph{Sencha Architect} per la realizzazione di una \ac{GUI}: si è rivelato essere uno strumento essenziale per disegnare velocemente ed efficacemente interfacce complesse, fornendo un aiuto indispensabile allo sviluppatore meno esperto.

Nel complesso l'azienda è rimasta soddisfatta del lavoro svolto, sia per la realizzazione dei prototipi \emph{MyNotes} e \emph{SensorDevice}, sia per la produzione della documentazione completa del codice sorgente delle due applicazioni, realizzata mediante \emph{JSDuck} \cite{sencha:jsduck}.

Inoltre è stata prodotta la documentazione del componente \emph{SyncEngine}, integrando quanto svolto dallo stagista che lo ha sviluppato, in modo da fornire all'azienda un formato unico e uniforme per la consultazione del materiale prodotto, visibile in figura \ref{fig:documentazione syncengine}.

\begin{figure}[htb]
\centering
\includegraphics[scale=0.35]{gfx/screenshot/doc_syncEngine}
\caption{Screenshot documentazione \emph{SyncEngine}}
\label{fig:documentazione syncengine}
\end{figure}

\section{Criticità riscontrate}
L'analisi dei requisiti e la progettazione delle applicazioni non hanno generato problemi rilevanti, sia per la natura esplorativa dello stage sia per il supporto ricevuto dal tutor aziendale.

Le difficoltà maggiori si sono avute nella corretta interpretazione della documentazione \emph{Sencha} che in molti punti si è rivelata essere molto carente e imprecisa, lasciando ai commenti degli utilizzatori il compito di chiarire dubbi e inesattezze.

Anche la consultazione del forum ufficiale spesso è risultata vana in quanto, davanti alla richieste d'aiuto, gli sviluppatori e i progettisti \emph{Sencha} non sempre riescono a dare soluzioni efficaci, lasciando in sospeso le conversazioni o dando risposte molto vaghe e poco utili nella pratica.

Un'ulteriore problema si è avuto con l'integrazione dei due framework: le guide trovate in Internet sono state utili per capire il singolo funzionamento dei framework ma, facendo riferimento a vecchie versioni degli stessi e risultando in alcuni casi anche in contrasto tra loro, sono state poco utili per la soluzione al problema: sono serviti diversi tentativi per comprendere la corretta tecnica.

L'arrivo di \emph{Sencha Touch 2.3} ha risolto parzialmente il problema creando degli \emph{adapter} verso le \ac{API} di \emph{Apache Cordova}; tali \emph{adapter} riflettono le funzionalità descritte in tabella \ref{tab:funzionalità sensordevice}. Purtroppo alcuni di questi, come \emph{Capture} e \emph{File}, presentavano, al tempo dello stage, dei bug che non hanno permesso di testare positivamente le relative caratteristiche.

\section{Conoscenze acquisite}
L'attività di stage mi ha dato la possibilità di vivere una realtà lavorativa in un'azienda solida, seguito da personale qualificato, facendomi mettere in pratica, anche se solo per un breve periodo, le conoscenze che ho acquisito attraverso gli studi universitari e dandomi la possibilità di approcciare la programmazione per sistemi mobili cogliendone alcuni pregi e difetti.

Mi ha permesso di studiare \emph{Sencha Touch} e \emph{Apache Cordova}: i due framework si sono rivelati essere strumenti molto interessanti per lo sviluppo di applicazioni mobile multi-piattaforma; se utilizzati insieme, infatti, danno la possibilità ad uno sviluppatore di realizzare software che sfrutti in modo ottimale i sensori dei device e allo stesso tempo  di donare alle applicazioni un aspetto grafico gradevole e che può essere reso accattivante e personalizzato con poche modifiche.

Infine ho avuto la possibilità di seguire interamente e in prima persona tutte le attività di realizzazione di un prodotto software, dall'analisi iniziale, passando per la progettazione e concludendo con lo sviluppo e i test, dovendo prendere in modo autonomo e responsabile decisioni importanti che mi consentissero di portare a termine con successo il progetto affidatomi.

Mi ritengo quindi molto soddisfatto dell'esperienza vissuta perché mi ha permesso di confrontarmi con le difficoltà di un progetto reale a cui sono riuscito a trovare delle soluzioni efficaci mettendo in pratica le mie capacità.