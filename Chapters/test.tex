%************************************************
\chapter{Test}\label{ch:test}
%************************************************
Il seguente capitolo descrive i test funzionali che sono stati svolti sui prototipi per verificarne il corretto funzionamento e la corrispondenza con i casi d'uso e i requisiti emersi dall'analisi effettuata.

Tali test sono stati eseguiti numerose volte e si sono svolti mantenendo collegato il dispositivo al notebook mediante cavo USB, con lo scopo di sfruttare la console di debug fornita da \emph{Eclipse} per monitorare il funzionamento dei prototipi.
Infatti ad ogni metodo presente nei prototipi degli oggetti sviluppati è stato inserito almeno un comando di output a console per identificare il flusso del programma.
Questa tecnica ha consentito di individuare alcuni malfunzionamenti e di comprendere meglio gli oggetti ritornati da alcune chiamate di metodi propri dei framework utilizzati, i quali erano documentati in modo carente o erroneo.

I test svolti presentano un codice\marginpar{Codifica test} identificativo composto dal prefisso \emph{T} seguito da un codice numerico progressivo.

\section{MyNotes}
\subsection{T1 - Visualizzazione note}
\begin{description}
\item[descrizione]: Dopo aver avviato con successo l'applicazione, l'utente deve vedere una lista delle note esistenti; nel caso che l'applicazione venga avviata per la prima volta, la lista deve risultare vuota.
\item[requisito verificato]: R0F1, R0F1.1
\item[esito]: \emph{superato}
\end{description}

\subsection{T2 - Creazione nuova nota}
\begin{description}
\item[descrizione]: \hfill
	\begin{enumerate}
	\item L'utente seleziona la creazione di una nuova nota e il sistema visualizza la pagina di compilazione;
	\item L'utente compila i campi relativi alla nuova nota;
	\item L'utente sceglie di salvare la nota e il sistema la salva correttamente e la visualizza nella lista;
	\end{enumerate}
\item[requisiti verificati]: R0F1, R0F1.2, R0F1.2.1, R0F1.2.2
\item[esito]: \emph{superato}
\end{description}

\subsection{T3 - Eliminazione nuova nota}
\begin{description}
\item[descrizione]: \hfill
	\begin{enumerate}
	\item L'utente seleziona la creazione di una nuova nota e il sistema visualizza la pagina di compilazione;
	\item L'utente compila i campi relativi alla nuova nota;
	\item L'utente sceglie di eliminare la nuova nota e il sistema chiede conferma di eliminazione;
	\item L'utente conferma la cancellazione della nota e il sistema elimina la nuova nota;
	\end{enumerate}
\item[requisiti verificati]: R0F1, R0F1.2, R0F1.2.1, R0F1.2.3, R0F1.2.4, R0F1.2.4.1
\item[esito]: \emph{superato}
\end{description}

\subsection{T4 - Annullamento eliminazione nuova nota}
\begin{description}
\item[descrizione]: \hfill
	\begin{enumerate}
	\item L'utente seleziona la creazione di una nuova nota e il sistema visualizza la pagina di compilazione;
	\item L'utente compila i campi relativi alla nuova nota;
	\item L'utente sceglie di eliminare la nuova nota e il sistema chiede conferma di eliminazione;
	\item L'utente annulla la cancellazione della nota e il sistema annulla l'eliminazione;
	\end{enumerate}
\item[requisiti verificati]: R0F1, R0F1.2, R0F1.2.1, R0F1.2.3, R0F1.2.4, R0F1.2.4.2
\item[esito]: \emph{superato}
\end{description}

\subsection{T5 - Modifica nota esistente}
\begin{description}
\item[descrizione]: \hfill
	\begin{enumerate}
	\item L'utente seleziona una nota esistente dalla lista e il sistema visualizza la pagina di modifica;
	\item L'utente effettua le modifiche desiderate;
	\item L'utente sceglie di salvare le modifiche apportate alla nota e il sistema la salva correttamente e la visualizza nella lista;
	\end{enumerate}
\item[requisiti verificati]: R0F1, R0F1.3, R0F1.3.1, R0F1.3.2
\item[esito]: \emph{superato}
\end{description}

\subsection{T6 - Eliminazione nota esistente}
\begin{description}
\item[descrizione]: \hfill
	\begin{enumerate}
	\item L'utente seleziona una nota esistente dalla lista e il sistema visualizza la pagina di modifica;
	\item L'utente sceglie di eliminare la nota e il sistema chiede conferma di eliminazione;
	\item L'utente conferma la cancellazione della nota e il sistema elimina la nota esistente;
	\end{enumerate}
\item[requisiti verificati]: R0F1, R0F1.3, R0F1.3.3, R0F1.3.4, R0F1.3.4.1
\item[esito]: \emph{superato}
\end{description}

\subsection{T7 - Annullamento eliminazione nota esistente}
\begin{description}
\item[descrizione]: \hfill
	\begin{enumerate}
	\item L'utente seleziona la creazione di una nota esistente e il sistema visualizza la pagina di modifica;
	\item L'utente sceglie di eliminare la nota e il sistema chiede conferma di eliminazione;
	\item L'utente annulla la cancellazione della nota e il sistema annulla l'eliminazione;
	\end{enumerate}
\item[requisiti verificati]: R0F1, R0F1.3, R0F1.3.3, R0F1.3.4, R0F1.3.4.2
\item[esito]: \emph{superato}
\end{description}

\subsection{T8 - Visualizzazione autori}
\begin{description}
\item[descrizione]: Dopo aver avviato con successo l'applicazione, l'utente deve visualizzare la lista degli autori esistenti; nel caso sia il primo avvio dell'applicazione o che non siano stati ancora creati degli autori, la lista deve risultare vuota.
\item[requisiti verificati]: R0F2, R0F2.1
\item[esito]: \emph{superato}
\end{description}

\subsection{T9 - Creazione nuovo autore}
\begin{description}
\item[descrizione]: \hfill
	\begin{enumerate}
	\item L'utente seleziona la creazione di un nuovo autore e il sistema visualizza la pagina di compilazione;
	\item L'utente compila i campi relativi alla nuovo autore;
	\item L'utente sceglie di salvare l'autore e il sistema lo salva correttamente e lo visualizza nella lista;
	\end{enumerate}
\item[requisiti verificati]: R0F2, R0F2.2, R0F2.2.1, R0F2.2.2
\item[esito]: \emph{superato}
\end{description}

\subsection{T10 - Eliminazione nuovo autore}
\begin{description}
\item[descrizione]: \hfill
	\begin{enumerate}
	\item L'utente seleziona la creazione di un nuova autore e il sistema visualizza la pagina di compilazione;
	\item L'utente compila i campi relativi al nuovo autore;
	\item L'utente sceglie di eliminare il nuovo autore e il sistema chiede conferma di eliminazione;
	\item L'utente conferma la cancellazione dell'autore e il sistema elimina il nuovo autore;
	\end{enumerate}
\item[requisiti verificati]: R0F2, R0F2.2, R0F2.2.1, R0F2.2.3, R0F2.2.4, R0F2.2.4.1
\item[esito]: \emph{superato}
\end{description}

\subsection{T11 - Annullamento eliminazione nuovo autore}
\begin{description}
\item[descrizione]: \hfill
	\begin{enumerate}
	\item L'utente seleziona la creazione di un nuovo autore e il sistema visualizza la pagina di compilazione;
	\item L'utente compila i campi relativi al nuovo autore;
	\item L'utente sceglie di eliminare il nuovo autore e il sistema chiede conferma di eliminazione;
	\item L'utente annulla la cancellazione dell'autore e il sistema annulla l'eliminazione;
	\end{enumerate}
\item[requisiti verificati]: R0F2, R0F2.2, R0F2.2.1, R0F2.2.3, R0F2.2.4, R0F2.2.4.2
\item[esito]: \emph{superato}
\end{description}

\subsection{T12 - Modifica autore esistente}
\begin{description}
\item[descrizione]: \hfill
	\begin{enumerate}
	\item L'utente seleziona un autore esistente dalla lista e il sistema visualizza la pagina di modifica;
	\item L'utente effettua le modifiche desiderate;
	\item L'utente sceglie di salvare le modifiche apportate all'autore e il sistema lo salva correttamente e lo visualizza nella lista;
	\end{enumerate}
\item[requisiti verificati]: R0F2, R0F2.3, R0F2.3.1, R0F2.3.2
\item[esito]: \emph{superato}
\end{description}

\subsection{T13 - Eliminazione autore esistente}
\begin{description}
\item[descrizione]: \hfill
	\begin{enumerate}
	\item L'utente seleziona un autore esistente dalla lista e il sistema visualizza la pagina di modifica;
	\item L'utente sceglie di eliminare l'autore e il sistema chiede conferma di eliminazione;
	\item L'utente conferma la cancellazione dell'autore e il sistema elimina l'autore esistente;
	\end{enumerate}
\item[requisiti verificati]: R0F2, R0F2.3, R0F2.3.3, R0F2.3.4, R0F2.3.4.1
\item[esito]: \emph{superato}
\end{description}

\subsection{T14 - Annullamento eliminazione autore esistente}
\begin{description}
\item[descrizione]: \hfill
	\begin{enumerate}
	\item L'utente seleziona la creazione di un autore esistente e il sistema visualizza la pagina di modifica;
	\item L'utente sceglie di eliminare l'autore e il sistema chiede conferma di eliminazione;
	\item L'utente annulla la cancellazione dell'autore e il sistema annulla l'eliminazione;
	\end{enumerate}
\item[requisiti verificati]: R0F2, R0F2.3, R0F2.3.3, R0F2.3.4, R0F2.3.4.2
\item[esito]: \emph{superato}
\end{description}

\subsection{T15 - Download dati dal server}
\begin{description}
\item[descrizione]: Dopo aver avviato con successo l'applicazione, l'utente decide di effettuare il download dei dati presenti nel server; il sistema si collega al server, scarica i dati e li unisce con quelli presenti nel dispositivo.
Nel caso non siano presenti dati nel server, l'utente non nota nessuna modifica alla lista delle note e degli autori.
\item[requisiti verificati]: R0F3, R0F3.1
\item[esito]: \emph{superato}
\end{description}

\subsection{T16 - Upload dati sul server}
\begin{description}
\item[descrizione]: Dopo aver avviato con successo l'applicazione, l'utente decide di effettuare l'upload sul server dei dati presenti nel dispositivo; il sistema prepara i dati e li spedisce al server.
Nel caso non siano presenti dati nel dispositivo, il sistema invia un pacchetto vuoto.
\item[requisiti verificati]: R0F3, R0F3.2
\item[esito]: \emph{superato}
\end{description}

\subsection{T17 - Gestione informazioni del dispositivo}
\begin{description}
\item[descrizione]: \hfill
	\begin{itemize}
	\item Dopo aver avviato con successo l'applicazione, l'utente decide di compilare le informazioni del dispositivo, inserendo un nome per il device e una descrizione;
	\item L'utente sceglie di salvare le informazioni immesse e il sistema le salva correttamente.
	\end{itemize}
\item[requisiti verificati]: R0F4, R0F4.1, R0F4.2
\item[esito]: \emph{superato}
\end{description}

\subsection{Tracciamento requisiti - test}
Di seguito viene riportata la tabella che traccia i singoli requisti dell'applicazione ai test effettuati che li soddisfano.
%**************************************
%    tabella tracciamento requisiti test MyNotes
%**************************************
\rowcolors{1}{lightCyan}{paleTurquoise}
\begin{longtable}{ll}
\hiderowcolors
\caption{Tracciamento requisiti - test -- MyNotes}
\label{tab:tracciamento requisiti-test mynotes} \\
%intestazione iniziale
\toprule \hiderowcolors
Requisito & Test\\
\midrule
\endfirsthead
%intestazione normale
\hiderowcolors
\multicolumn{2}{l}{\footnotesize\itshape Continua dalla pagina precedente}\\
\toprule \hiderowcolors
Requisito & Test\\
\midrule
\endhead
%piede normale
\midrule \hiderowcolors
\multicolumn{2}{r}{\footnotesize\itshape Continua nella prossima pagina}\\
\endfoot
%piede finale
\bottomrule \hiderowcolors
\multicolumn{2}{r}{\footnotesize\itshape Si conclude dalla pagina precedente}\\
\endlastfoot
%corpo della tabella
\showrowcolors 
R0F1			& T1, T2, T3, T4, T5, T6, T7 \\
R0F1.1			& T1 \\
R0F1.2			& T2, T3, T4 \\
R0F1.2.1		& T2, T3, T4 \\
R0F1.2.2		& T2 \\
R0F1.2.3		& T3, T4 \\
R0F1.2.4		& T3, T4 \\
R0F1.2.4.1		& T3 \\
R0F1.2.4.2		& T4 \\
R0F1.3			& T5, T6, T7 \\
R0F1.3.1		& T5 \\
R0F1.3.2		& T5 \\
R0F1.3.3		& T6, T7 \\
R0F1.3.4		& T6, T7 \\
R0F1.3.4.1		& T6 \\
R0F1.3.4.2		& T7 \\
R0F2			& T8, T9, T10, T11, T12, T13, T14 \\
R0F2.1			& T8 \\
R0F2.2			& T9, T10, T11 \\
R0F2.2.1		& T9, T10, T11 \\
R0F2.2.2		& T9 \\
R0F2.2.3		& T10, T11 \\
R0F2.2.4		& T10, T11 \\
R0F2.2.4.1		& T10 \\
R0F2.2.4.2		& T11 \\
R0F2.3			& T12, T13, T14 \\
R0F2.3.1		& T12 \\
R0F2.3.2		& T12 \\
R0F2.3.3		& T13, T14 \\
R0F2.3.4		& T13, T14 \\
R0F2.3.4.1		& T13 \\
R0F2.3.4.2		& T14 \\
R0F3			& T15, T16 \\
R0F3.1			& T15 \\
R0F3.2			& T16 \\
R0F4			& T17 \\
R0F4.1			& T17 \\
R0F4.2			& T17 \\
\end{longtable}

\section{SensorDevice}
\subsection{T1 - Lettura barcode}
\begin{description}
\item[descrizione]: \hfill
	\begin{itemize}
	\item Dopo aver avviato con successo l'applicazione, l'utente sceglie di effettuare la lettura di un codice a barre;
	\item Il sistema accede alla fotocamera e permette all'utente di inquadrare un codice a barre;
	\item L'utente inquadra il barcode e il sistema lo legge correttamente, chiude la fotocamera, salva il codice letto e lo visualizza.
	\end{itemize}
\item[requisito verificato]: R1F1, R1F9, R1F10
\item[esito]: \emph{superato}
\end{description}

\subsection{T2 - Scatto fotografia}
\begin{description}
\item[descrizione]: \hfill
	\begin{itemize}
	\item Dopo aver avviato con successo l'applicazione, l'utente sceglie di scattare una fotografia;
	\item Il sistema accede alla fotocamera e permette all'utente di scattare una fotografia;
	\item L'utente scatta la fotografia e il sistema chiude la fotocamera e salva l'immagine scattata.
	\end{itemize}
\item[requisito verificato]: R1F2, R1F2.1, R1F2.1.1, R1F10
\item[esito]: \emph{superato}
\end{description}

\subsection{T3 - Recupero immagine da galleria}
\begin{description}
\item[descrizione]: \hfill
	\begin{itemize}
	\item Dopo aver avviato con successo l'applicazione, l'utente sceglie di recuperare un'immagine esistente dalla galleria;
	\item Il sistema accede alla galleria del dispositivo e permette all'utente di selezionare un immagine;
	\item L'utente seleziona l'immagine da importare e il sistema chiude la galleria e salva l'immagine selezionata.
	\end{itemize}
\item[requisito verificato]: R1F2, R1F2.1, R1F2.1.2, R1F10
\item[esito]: \emph{superato}
\end{description}

\subsection{T4 - Cattura video}
\begin{description}
\item[descrizione]: \hfill
	\begin{itemize}
	\item Dopo aver avviato con successo l'applicazione, l'utente sceglie di effettuare la cattura di un video;
	\item Il sistema accede alla fotocamera e permette all'utente di registrare un video;
	\item L'utente realizza un video e il sistema chiude la fotocamera e salva il video realizzato.
	\end{itemize}
\item[requisito verificato]: R1F2, R1F2.2, R1F10
\item[esito]: \emph{superato}
\end{description}

\subsection{T5 - Cattura audio}
\begin{description}
\item[descrizione]: \hfill
	\begin{itemize}
	\item Dopo aver avviato con successo l'applicazione, l'utente sceglie di effettuare la cattura di un file audio;
	\item Il sistema accede al microfono e permette all'utente di registrare una traccia audio;
	\item L'utente registra una traccia audio e il sistema chiude il microfono e salva la traccia realizzata.
	\end{itemize}
\item[requisito verificato]: R1F2, R1F2.3, R1F10
\item[esito]: \emph{superato}
\end{description}

\subsection{T6 - Recupero informazioni dispositivo}
\begin{description}
\item[descrizione]: \hfill
	\begin{itemize}
	\item Dopo aver avviato con successo l'applicazione, l'utente sceglie di recuperare le informazioni proprie del dispositivo;
	\item Il sistema accede al dispositivo e recupera il modello del dispositivo, il sistema operativo installato e la versione, l'\ac{UUID} e la versione di \emph{Cordova} in uso e li visualizza.
	\end{itemize}
\item[requisito verificato]: R1F3, R1F9, R1F10
\item[esito]: \emph{superato}
\end{description}

\subsection{T7 - Recupero posizione corrente}
\begin{description}
\item[descrizione]: \hfill
	\begin{itemize}
	\item Dopo aver avviato con successo l'applicazione, l'utente sceglie di recuperare la posizione corrente tramite il \ac{GPS};
	\item Il sistema accede al ricevitore \ac{GPS} del dispositivo, recupera la posizione corrente, la salva e la visualizza.
	\end{itemize}
\item[requisito verificato]: R1F4, R1F9, R1F10
\item[esito]: \emph{superato}
\end{description}

\subsection{T8 - Recupero contatti rubrica}
\begin{description}
\item[descrizione]: \hfill
	\begin{itemize}
	\item Dopo aver avviato con successo l'applicazione, l'utente sceglie di recuperare i contatti della rubrica presenti nel dispositivo;
	\item Il sistema accede alla rubrica del dispositivo e recupera i contatti presenti, li salva e li visualizza.
	\end{itemize}
\item[requisito verificato]: R1F5, R1F9, R1F10
\item[esito]: \emph{superato}
\end{description}

\subsection{T9 - Salvataggio informazioni personali utente}
\begin{description}
\item[descrizione]: \hfill
	\begin{itemize}
	\item Dopo aver avviato con successo l'applicazione, l'utente sceglie di compilare le informazioni personali accedendo alla pagina dedicata;
	\item L'utente decide di salvare le informazioni immesse e il sistema le salva correttamente.
	\end{itemize}
\item[requisito verificato]: R1F6, R1F6.1, R1F6.2, R1F9, R1F10
\item[esito]: \emph{superato}
\end{description}

\subsection{T10 - Eliminazione informazioni personali utente}
\begin{description}
\item[descrizione]: \hfill
	\begin{itemize}
	\item Dopo aver avviato con successo l'applicazione, l'utente sceglie di eliminare le informazioni personali precedentemente salvate;
	\item Il sistema elimina le informazioni salvate.
	\end{itemize}
\item[requisito verificato]: R1F6, R1F6.3, R1F9, R1F10
\item[esito]: \emph{superato}
\end{description}

\subsection{T11 - Backup informazioni personali utente}
\begin{description}
\item[descrizione]: \hfill
	\begin{itemize}
	\item Dopo aver avviato con successo l'applicazione, l'utente sceglie di effettuare il backup delle informazioni personali salvate sul dispositivo;
	\item Il sistema effettua il salvataggio sulla memoria di massa del dispositivo in un file di testo presente nel percorso \code{/sdcard/SensorDevice/backpPersonalInfo.txt}.
	\end{itemize}
\item[requisito verificato]: R1F6, R1F6.4
\item[esito]: \emph{superato}
\end{description}

\subsection{T12 - Ripristino informazioni personali utente}
\begin{description}
\item[descrizione]: \hfill
	\begin{itemize}
	\item Dopo aver avviato con successo l'applicazione, l'utente sceglie di effettuare il ripristino delle informazioni personali di cui è stato precedentemente effettuato un backup;
	\item Il sistema recupera le informazioni dal file presente nel percorso \code{/sdcard/SensorDevice/backpPersonalInfo.txt} e le salva nello store dedicata sostituendo quelle presenti.
	\end{itemize}
\item[requisito verificato]: R1F6, R1F6.5
\item[esito]: \emph{superato}
\end{description}

\subsection{T13 - Consultazione galleria immagini}
\begin{description}
\item[descrizione]: Dopo aver avviato con successo l'applicazione, l'utente sceglie di consultare la galleria immagini accedendo alla pagina dedicata e il sistema visualizza le immagini presenti; nel caso sia il primo avvio dell'applicazione o non siano presenti immagini la galleria deve risultare vuota.
\item[requisito verificato]: R1F7
\item[esito]: \emph{superato}
\end{description}

\subsection{T14 - Consultazione libreria audio/video}
\begin{description}
\item[descrizione]: Dopo aver avviato con successo l'applicazione, l'utente sceglie di consultare la libreria audio/video accedendo alla pagina dedicata e il sistema visualizza un elenco delle registrazioni audio e video presenti; nel caso sia il primo avvio dell'applicazione o non siano presenti registrazioni la libreria deve risultare vuota.
\item[requisito verificato]: R1F8
\item[esito]: \emph{superato}
\end{description}

\subsection{Tracciamento requisiti - test}
Di seguito viene riportata la tabella che traccia i singoli requisti dell'applicazione ai test effettuati che li soddisfano.
%**************************************
%    tabella tracciamento requisiti test SensorDevice
%**************************************
\rowcolors{1}{lightCyan}{paleTurquoise}
\begin{longtable}{ll}
\hiderowcolors
\caption{Tracciamento requisiti - test -- SensorDevice}
\label{tab:tracciamento requisiti-test sensordevice} \\
%intestazione iniziale
\toprule \hiderowcolors
Requisito & Test\\
\midrule
\endfirsthead
%intestazione normale
\hiderowcolors
\multicolumn{2}{l}{\footnotesize\itshape Continua dalla pagina precedente}\\
\toprule \hiderowcolors
Requisito & Test\\
\midrule
\endhead
%piede normale
\midrule \hiderowcolors
\multicolumn{2}{r}{\footnotesize\itshape Continua nella prossima pagina}\\
\endfoot
%piede finale
\bottomrule \hiderowcolors
%\multicolumn{2}{r}{\footnotesize\itshape Si conclude dalla pagina precedente}\\
\endlastfoot
%corpo della tabella
\showrowcolors 
R1F1			& T1 \\
R1F2			& T2, T3, T4, T5 \\
R1F2.1			& T2, T3 \\
R1F2.1.1		& T2 \\
R1F2.1.2		& T3 \\
R1F2.2			& T4 \\
R1F2.3			& T5 \\
R1F3			& T6 \\
R1F4			& T7 \\
R1F5			& T8 \\
R1F6			& T9, T10, T11, T12 \\
R1F6.1			& T9 \\
R1F6.2			& T9 \\
R1F6.3			& T10 \\
R1F6.4			& T11 \\
R1F6.5			& T12 \\
R1F7			& T13 \\
R1F8			& T14 \\
R1F9			& T1, T6, T7, T8, T9, T10 \\
R1F10			& T1, T2, T3, T4, T5, T6, T7, T8, T9, T10 \\
\end{longtable} 
