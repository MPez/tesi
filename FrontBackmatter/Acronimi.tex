%*******************************************************
% Acronyms
%*******************************************************
%\phantomsection 
\refstepcounter{dummy}
\pdfbookmark[1]{Acronimi}{Acronimi}
\markboth{\spacedlowsmallcaps{Acronimi}}{\spacedlowsmallcaps{Acronimi}}
\section*{Acronimi}
\begin{acronym}[DBMS]
	\acro{BYOD}{Bring Your Own Device}
	
	{\small È un'espressione per riferirsi alle politiche aziendali che permettono di portare i propri dispositivi personali nel posto di lavoro e usarli per avere gli accessi privilegiati alle informazioni aziendali e alle loro applicazioni. Il termine è anche usato per descrivere le stesse pratiche applicati agli studenti che usano i loro dispositivi in ambito educativo. \par}

	\acro{SCM}{Supply Chain Management}
	
	{\small Sistema di gestione di diverse attività logistiche delle aziende, con l'obiettivo di controllare le prestazioni e migliorarne l'efficienza. Tra queste attività sono incluse la catalogazione sistematica dei prodotti e il coordinamento strategico dei vari membri della catena di distribuzione. \par}

	\acro{CRM}{Customer Relationship Management}
	
	{\small Sistema di gestione delle relazioni aziendali coi clienti, necessario ad un impresa market-oriented per garantire un efficace attuazione del proprio management. \par}

	\acro{ERP}{Enterprise Resource Planning}
	
	{\small Sistema di gestione, chiamato in informatica sistema informativo, che integra tutti i processi di business rilevanti di un'azienda (vendite, acquisti, gestione magazzino, contabilità etc.). \par}

	\acro{DBMS}{Database Management System}
	
	{\small  Sistema software progettato per consentire la creazione, la manipolazione e l'interrogazione efficiente  di database. \par}

	\acro{PMI}{Piccole e Medie Imprese}
    	
	{\small Le piccole e medie imprese o PMI sono aziende le cui dimensioni rientrano entro certi limiti occupazionali e finanziari prefissati. \par}
    	
	\acro{API}{Application Programming Interface}
	
	{\small Insieme di procedure disponibili al programmatore, di solito raggruppate a formare un set di strumenti specifici per l'espletamento di un determinato compito all'interno di un certo programma. \par}
	
	\acro{UML}{Unified Modeling Language}
	
	{\small  Linguaggio di modellazione e specifica basato sul paradigma object-oriented. \par}
\end{acronym}                      
