%*******************************************************
% Acronyms
%*******************************************************
%\phantomsection 
\refstepcounter{dummy}
\pdfbookmark[1]{Acronimi}{Acronimi}
\markboth{\spacedlowsmallcaps{Acronimi}}{\spacedlowsmallcaps{Acronimi}}
\chapter*{Acronimi}
\begin{acronym}[HTML5]
	\acro{ADT}{Android Developer Tools}
	
	{\small Plugin per Eclipse che fornisce una serie di strumenti che aiutano lo sviluppatore a generare velocemente un'applicazione Android. \par}
	
	\acro{API}{Application Programming Interface}
	
	{\small Insieme di procedure disponibili al programmatore, di solito raggruppate a formare un set di strumenti specifici per l'espletamento di un determinato compito all'interno di un certo programma. \par}
	
	\acro{APK}{Android PacKage}
	
	{\small Questo formato di file, una variante del formato \ac{JAR}, è utilizzato per la distribuzione e l'installazione di componenti in dotazione sulla piattaforma per dispositivi mobili Android.  \par}
	
	\acro{ASF}{Apache Software Foundation}
	
	{\small È una fondazione no-profit ed una comunità di sviluppo di progetti software tra i quali il web server Apache. L'ASF fu costituita nel giugno 1999. L'Apache Software Foundation è una comunità distribuita di sviluppatori che lavorano su progetti software open source. Questi progetti sono caratterizzati da un processo di sviluppo distribuito, collaborativo e basato sul consenso molto simile al progetto wikipedia. Ciascun progetto è gestito da un team di volontari che sono i contributori attivi al progetto. L'ASF è, inoltre, meritocratica, perché l'appartenenza alla comunità è concessa solo a chi partecipa attivamente ai progetti. Tra gli obiettivi di ASF ci sono la protezione legale dei volontari partecipanti e la protezione del marchio Apache dall'uso fraudolento da parte di organizzazioni terze. I progetti ASF sono tutti licenziati con la Apache License. \par}
	
	\acro{BYOD}{Bring Your Own Device}
	
	{\small È un'espressione per riferirsi alle politiche aziendali che permettono di portare i propri dispositivi personali nel posto di lavoro e usarli per avere gli accessi privilegiati alle informazioni aziendali e alle loro applicazioni. Il termine è anche usato per descrivere le stesse pratiche applicati agli studenti che usano i loro dispositivi in ambito educativo. \par}

	\acro{CLI}{Command Line Interface}
	
	{\small Indica una tipologia di interfaccia utente caratterizzata da un'interazione di tipo testuale tra utente ed elaboratore: l'utente impartisce comandi testuali in input mediante tastiera alfanumerica e riceve risposte testuali in output dall'elaboratore mediante display o stampante alfanumerici. \par}
	
	\acro{CRM}{Customer Relationship Management}
	
	{\small Sistema di gestione delle relazioni aziendali coi clienti, necessario ad un impresa market-oriented per garantire un efficace attuazione del proprio management. \par}
	
	\acro{CSS3}{Cascading Style Sheets 3}
	
	{\small È un linguaggio usato per definire la formattazione di documenti HTML, XHTML e XML ad esempio in siti web e relative pagine web. \par}
	
	\acro{DBMS}{Database Management System}
	
	{\small  Sistema software progettato per consentire la creazione, la manipolazione e l'interrogazione efficiente  di database. \par}

	\acro{ERP}{Enterprise Resource Planning}
	
	{\small Sistema di gestione, chiamato in informatica sistema informativo, che integra tutti i processi di business rilevanti di un'azienda (vendite, acquisti, gestione magazzino, contabilità etc.). \par}
	
	\acro{GPS}{Global Positioning System}
	
	{\small È un sistema di posizionamento e navigazione satellitare civile che, attraverso una rete satellitare dedicata di satelliti artificiali in orbita, fornisce ad un terminale mobile o ricevitore GPS informazioni sulle sue coordinate geografiche ed orario, in ogni condizione meteorologica, ovunque sulla Terra o nelle sue immediate vicinanze ove vi sia un contatto privo di ostacoli con almeno quattro satelliti del sistema. La localizzazione avviene tramite la trasmissione di un segnale radio da parte di ciascun satellite e l'elaborazione dei segnali ricevuti da parte del ricevitore. \par}

	\acro{GUI}{Graphic User Interface}
	
	{\small È un tipo di interfaccia utente che consente all'utente di interagire con la macchina manipolando oggetti grafici convenzionali. Con l'espressione interfaccia grafica si indica l'interfaccia di un qualunque programma: dal sistema operativo, al software applicativo. \par}
	
	\acro{HTML}{HyperText Markup Language}
	
	{\small È linguaggio di markup solitamente usato per la formattazione di documenti ipertestuali disponibili nel World Wide Web sotto forma di pagine web. \par}
	
	\acro{HTML5}{HyperText Markup Language 5}
	
	{\small È la quinta versione di HTML e il suo sviluppo si è concentrato sulla separazione tra struttura e presentazione delle pagine web aggiungendo molte novità per il controllo di file multimediali, la geolocalizzazione, i canvas, il web storage. \par}

	\acro{IDE}{Integrated Development Environment}
	
	{\small È un software che, in fase di programmazione, aiuta i programmatori nello sviluppo del codice sorgente di un programma. Normalmente è uno strumento software che consiste di più componenti, da cui appunto il nome integrato: un editor di codice sorgente, un compilatore e/o un interprete, un tool di building automatico, un debugger. A volte è integrato anche con un sistema di controllo di versione e con uno o più tool per semplificare la costruzione di una GUI. \par}
	
	\acro{JAR}{Java ARchive}
	
	{\small Indica un archivio dati compresso (ZIP) usato per distribuire raccolte di classi Java. Tali file sono concettualmente e praticamente assimilabili a package, e quindi talvolta associabili al concetto di libreria. \par}	
	
	\acro{MVC}{Model View Controller}
	
	{\small È un pattern architetturale molto diffuso nello sviluppo di sistemi software, in particolare nell'ambito della programmazione orientata agli oggetti, in grado di separare la logica di presentazione dei dati dalla logica di business. \par}
	
	\acro{PMI}{Piccole e Medie Imprese}
	
	{\small Le piccole e medie imprese o PMI sono aziende le cui dimensioni rientrano entro certi limiti occupazionali e finanziari prefissati. \par}

	\acro{SCM}{Supply Chain Management}
	
	{\small Sistema di gestione di diverse attività logistiche delle aziende, con l'obiettivo di controllare le prestazioni e migliorarne l'efficienza. Tra queste attività sono incluse la catalogazione sistematica dei prodotti e il coordinamento strategico dei vari membri della catena di distribuzione. \par}

	\acro{SDK}{Software Development Kit}
	
	{\small Indica genericamente un insieme di strumenti per lo sviluppo e la documentazione di software. \par}

	\acro{UML}{Unified Modeling Language}
	
	{\small  Linguaggio di modellazione e specifica basato sul paradigma object-oriented. \par}
	
	\acro{UUID}{Universally Unique IDentifier}
	
	{\small È un identificativo standard usato nelle infrastrutture software, standardizzato dalla Open Software Foundation (OSF) come parte di un ambiente distribuito di computazione.\\
Lo scopo dell'UUID è di abilitare un sistema distribuito all'identificazione di informazioni in assenza di un sistema centralizzato di coordinamento. In questo contesto la chiave univoca dovrebbe implicare "l'univocità" pratica senza "garantirla". Il fatto che le chiavi siano in numero finito implica che due entità potrebbero avere la stessa chiave identificativa. In pratica, l'ampiezza dello spazio delle chiavi e il loro processo di generazione offrono sufficienti garanzie che la stessa chiave non venga assegnata a due entità differenti. Chiunque può creare un UUID e usarlo con ragionevole probabilità che non venga usato da nessun altro. Le informazioni associate all'UUID possono essere in seguito combinate in un singolo database senza necessità di dover risolvere eventuali conflitti. \par}
\end{acronym}
